% !TeX program = xelatex
% ↑ Automatische Auswahl für XeLaTeX compiler

% Das ist mein Template für die TX000 Arbeiten. Nicht perfekt, also falls ihr Verbesserungsvorschläge habt, stellt gerne einen Pull-Request. https://github.com/NikomitK/TX000_Template
% Bitte lasst auch einen star auf github da, danke.
% Wenn ihr die cite funktion von LaTeX nutzen wollt, müsst ihr einfach die Quellen im bibtex Format in die sources.bib datei kopieren, google scholar z.B. hat bei den Quellen einen Button mit dem man das so bekommt, auch viele andere websites.
% Bilder kommen in den images Ordner, den müsst ihr beim abrufen eines Bildes nicht angeben, passiert automatisch.


% Trag hier deine Daten ein, die entsprechenden Felder werden automatisch angepasst.
\def\meinTitel{Verteilte Systeme: Filesharing-Plattform}
\def\artDerArbeit{Ausarbeitung}
\def\meinName{Aziz Carducci, Sven Sendke}
\def\meinKurs{TINF22F}
\def\meineMatrikelNr{1965015, 8469950}
\def\firmenName{Allianz Lebensversicherungs-AG}
\def\projektBetreuer{Benjamin Salchow}
\def\abgabeDatum{16.12.2024}
%-----------------------------------------------------------------------------------


\documentclass[12pt]{report}
\usepackage[heightrounded]{geometry}
\geometry{
	a4paper,
	lmargin=3.1cm, %Seitenrand left
	tmargin=2.8cm, %Seitenrad top
	headsep=35pt %Abstand von Kopfzeile
}
\usepackage[onehalfspacing]{setspace}
\usepackage[compact]{titlesec}
\usepackage{cite} % Zitierungen
\usepackage{struktex} % Struktogramme
\usepackage{array} % kp hat chatgpt benutzt
\usepackage{longtable} % Tabelle über einen seitenumbruch
\usepackage[nohyperlinks, printonlyused]{acronym} % abkürzungsverzeichnis
\usepackage{fontspec}
\usepackage{blindtext} % LoremIpsum
\usepackage{fancyhdr} % Kopf- und Fußzeile
\usepackage[export]{adjustbox} % Bilder alignment
\usepackage{stfloats} % Tabular at bottom
%\usepackage[table]{xcolor} %Tabellen farben
\usepackage{tabularx} % Anpassbarere tabellen
\usepackage{easyReview} % Review anmerkungen

\usepackage[utf8]{inputenc} % this is needed for umlauts
\usepackage[ngerman]{babel} % this is needed for umlauts
\usepackage[T1]{fontenc}    % this is needed for correct output of umlauts in pdf


\usepackage{subcaption} % Für subfigures glaub


\usepackage{tikz} % Zum zeichnen
\usetikzlibrary{calc}
\usetikzlibrary{shapes.geometric, arrows}
\setmainfont[Scale=1.1]{Arial}

%--------------------Flowcharts--------------------
\tikzstyle{startstop} = [rectangle, rounded corners, minimum width=3cm, minimum height=1cm,text centered, draw=black, fill=red!30]
\tikzstyle{io} = [trapezium, trapezium left angle=70, trapezium right angle=110, minimum width=3cm, minimum height=1cm, text centered, draw=black, fill=blue!30]
\tikzstyle{process} = [rectangle, minimum width=3cm, minimum height=1cm, text centered, draw=black, fill=orange!30]
\tikzstyle{decision} = [diamond, minimum width=3cm, minimum height=1cm, text centered, draw=black, fill=green!30]
\tikzstyle{arrow} = [thick,->,>=stealth]
%--------------------------------------------------


%--------------------Codeblöcke--------------------
\usepackage{listings} %Für Codeblöcke
\usepackage{color} %Farben für Codeblöcke?
\definecolor{dkgreen}{rgb}{0,0.6,0}
\definecolor{gray}{rgb}{0.5,0.5,0.5}
\definecolor{mauve}{rgb}{0.58,0,0.82}

\lstset{
	language=Java,
	aboveskip=3mm,
	belowskip=3mm,
	showstringspaces=false,
	columns=flexible,
	basicstyle={\small\ttfamily},
	numbers=none,
	numberstyle=\tiny\color{gray},
	keywordstyle=\color{blue},
	commentstyle=\color{dkgreen},
	stringstyle=\color{mauve},
	breaklines=true,
	breakatwhitespace=true,
	tabsize=3
}
%-------------------------------------------------

\usepackage{graphicx} %Package für Bilder
\graphicspath{ {./images/} } %Ordner für Bilder

\sloppy % damit lange Wörter nicht über die Zeile hinausgeschrieben werden.

%--------------------Chapter Heading--------------------
\makeatletter
\def\@makechapterhead#1{%
	\vspace*{-20\p@}%
	{\parindent \z@ \raggedright \normalfont
		\ifnum \c@secnumdepth >\m@ne
		%\huge\bfseries \@chapapp\space \thechapter
		\Huge\bfseries \thechapter.\space%
		%\par\nobreak
		%\vskip 20\p@
		\fi
		\interlinepenalty\@M
		\Huge \bfseries #1\par\nobreak
		\vskip 20\p@
}}
\makeatother
\makeatletter
\def\@makeschapterhead#1{%
	\vspace*{-20\p@}%
	{\parindent \z@ \raggedright \normalfont
		%\huge\bfseries \@chapapp\space \thechapter
		\Huge\bfseries\space%
		%\par\nobreak
		%\vskip 20\p@
		\interlinepenalty\@M
		\Huge \bfseries #1\par\nobreak
		\vskip 20\p@
}}
\makeatother
%-------------------------------------------------------

%\addto\captionsngerman{\renewcommand{\listfigurename}{}}
 % titel von abbildungsverzeichnis weg?

\setlength\parindent{0pt} %Auto Einrücken deaktivieren


%-------------Setup für Inhaltsverzeichnis--------------
\renewcommand{\contentsname}{Inhaltsverzeichnis} %Umbenennung TOC

\usepackage{tocloft} % Formatierung TOX

\setlength{\cftbeforetoctitleskip}{0pt}

\setlength{\cftsubsecnumwidth}{4em} %abstand kapitelnummer - titel

\renewcommand{\cfttoctitlefont}{\huge\bfseries}
\renewcommand{\cftloftitlefont}{\huge\bfseries}

\renewcommand\cftchapfont{\Large\bfseries}
\renewcommand\cftchappagefont{\large}

\renewcommand\cftsecfont{\large\bfseries}
\renewcommand\cftsecpagefont{\large}

\renewcommand\cftsubsecfont{\large}
\renewcommand\cftsubsecpagefont{\large}

\renewcommand\cftsubsubsecfont{\normalsize}
\renewcommand\cftsubsubsecpagefont{\normalsize}


\renewcommand\cftchapafterpnum{\par\addvspace{8pt}}
\renewcommand\cftsecafterpnum{\par\addvspace{8pt}}
\renewcommand\cftsubsecafterpnum{\par\addvspace{6pt}}
\renewcommand\cftsubsubsecafterpnum{\par\addvspace{6pt}}
%-------------------------------------------------------


%------------------Setup für LoF/LoT--------------------
\makeatletter
\renewcommand{\@cftmakeloftitle}{}
\renewcommand{\@cftmakelottitle}{}
\makeatother
\setlength{\cftfigindent}{0em} % change indentation of e.g. "Figure 1" within list of figures
\renewcommand\cftfigfont{\large}
\renewcommand\cftfigpagefont{\large}
\setlength{\cfttabindent}{0em} % change indentation of e.g. "Figure 1" within list of figures
\renewcommand\cfttabfont{\large}
\renewcommand\cfttabpagefont{\large}
\setlength{\cftbeforeloftitleskip}{0pt}
\setlength{\cftbeforelottitleskip}{0pt}
%-------------------------------------------------------


%----------------Setup für Verlinkungen-----------------
\usepackage{hyperref}
\hypersetup{
	colorlinks,
	citecolor=black,
	filecolor=black,
	linkcolor=black,
	urlcolor=black
}
%-------------------------------------------------------


%-------------Kopf-/Fußzeile für Titlepage--------------
\fancypagestyle{titlepage}
{
	\fancyhead[L]{\includegraphics[scale=0.09]{firmenlogo}}
	\fancyhead[R]{\includegraphics[scale=0.25]{dhbw}}
	\renewcommand{\headrulewidth}{0pt}
	\fancyfoot[C]{}
}
%-------------------------------------------------------

\begin{document} 
	
	\begin{titlepage}
		\thispagestyle{titlepage}
		\newcommand\HRule{\rule{\textwidth}{1pt}} %Titellinien
		
		
		\begin{center}
			
			\vspace*{2cm}
			
			%Title
			\begin{spacing}{2}
				{ \huge \bfseries \MakeUppercase{\meinTitel}}
				%{ \large \bfseries subTitle}\\[0.4cm]
			\end{spacing}
			
			\vspace*{1.5cm}
			
			%Art der Arbeit
			\Large \artDerArbeit
			
			\vspace*{2cm}
			
			%Hochschule
			{\LARGE Studiengang Informatik}\\
			{\LARGE an der Dualen Hochschule}\\
			{\LARGE Baden-Württemberg Stuttgart}\\
			
			\vspace*{2cm}
			
			\Large von \meinName
			
			\vspace*{1.5cm}
			
			\Large Abgabedatum: \abgabeDatum
			
			\begin{table*}[bp]
				\centering
				\begin{tabular}{l l}
					Kurs: & \meinKurs  \\
					Matrikelnummer: & \meineMatrikelNr  \\
					Unternehmen: & \firmenName \\
					Projektbetreuer: &  \projektBetreuer\\
				\end{tabular}
			\end{table*}
			
			
		\end{center}
		
	\end{titlepage}


%------------------Kopf- und Fußzeile-------------------
\fancypagestyle{plain}{
	\fancyfoot[L]{\meinName\\
		 \meinKurs}
	\fancyfoot[C]{Seite \thepage\ }% von \pageref{LastPage}}
	\fancyfoot[R]{\abgabeDatum}
}

\pagestyle{plain}
\fancyhead{}


\fancyhead[L]{\includegraphics[scale=0.09]{firmenlogo}}
\fancyhead[C]{\nouppercase\leftmark}
\fancyhead[R]{\includegraphics[scale=0.25]{dhbw}}

\renewcommand{\footrulewidth}{0.4pt} %Linie für Fußzeile


\renewcommand{\sectionmark}[1]{\markboth{#1}{}} 
%-------------------------------------------------------

\pagenumbering{Roman}
\newpage

%------------------Inhaltsverzeichnis-------------------

\addcontentsline{toc}{chapter}{\protect\numberline{}Inhaltsverzeichnis}
\tableofcontents
\addtocontents{toc}{}
\thispagestyle{plain}

%********************************
%Abbildungsverzeichnis
%********************************
\newpage
\chapter*{Abbildungsverzeichnis}
\addcontentsline{toc}{chapter}{\protect\numberline{}Abbildungsverzeichnis}

\listoffigures




%********************************
%Tabellenverzeichnis
%********************************
\newpage
\chapter*{Tabellenverzeichnis}
\addcontentsline{toc}{chapter}{\protect\numberline{}Tabellenverzeichnis}

\listoftables


\newpage
\chapter*{Abkürzungsverzeichnis}
\addcontentsline{toc}{chapter}{\protect\numberline{}Abkürzungsverzeichnis}
\markboth{Abkürzungsverzeichnis}{Abkürzungsverzeichnis} %Benötigt, damit das im header steht

%********************************
%Abkürzungsverzeichnis
%********************************
\begin{acronym}[SOAP]
	\acro{BA}{Beispiel Acronym}
\end{acronym}


%Speichern des page counters, um bei Literaturverzeichnis weiter zu zählen.
\newcounter{frontmatterPage}
\addtocounter{frontmatterPage}{\value{page}} 

\newpage
\pagenumbering{arabic}
\chapter{Architektur}
	Im Folgenden wird die Architektur im Allgemeinen mit den Systemkomponenten und den Anforderungen dargestellt.
	\section{Allgemein}
		Zuerst betrachten wir die Struktur der Architektur und dann die Architectual Desicion.
		\subsection{Aufbau}
			Auf der Abbildung ist die Allgemeine Architektur zu sehen \ref{fig:architektur}.
			\begin{figure}[h]
				\centering
				\includegraphics[width=\linewidth]{architektur}
				\caption{Architektur}
				\label{fig:architektur}
			\end{figure}
			
		\subsection{Architectual Decision}
			Microservice Architektur: 
			Deshalb Modularisierung
			• Aufteilung der Komponenten
			• ermöglicht Skalierung
			• Ausfallsicherheit
			
			Unabhängig skalierbar (File handler probably mehr skalieren als User handler)
			
			A key benefit of microservices is the maintainability. Breaking a system into independent and self-deployable services aids developer teams to test their services and make changes independently from other developers, simplifying distributed development. \cite{de2019monolithic}.
			
			Moreover, faulty microservices can be quickly restarted.  \cite{taibi2017processes}
			
			Minio:
			Finally, MinIO is able to deliver the high-performance object storage that is required by modern big data applications. \cite{makris2022performance}
			Overall, as indicated in the above figures, MinIO presents the best performance in both the read and write operations. \cite{makris2022performance}
			S3?
			
			Das Spring Boot Framework ist der De-facto-Standard für Java-Microservice-Architekturen. Es zeichnet sich durch seine Fähigkeit aus, enorme Datenmengen zu verarbeiten, was es besonders für die Big-Data-Industrie attraktiv macht \cite{mythily2022analysis}. Darüber hinaus erleichtert Spring Boot die Entwicklung von RESTful-Webservices und APIs erheblich, wodurch Entwickler ihre Aufgaben schneller und effizienter erledigen können \cite{mythily2022analysis}.
			
	\section{Systemkomponenten}
		Im Folgenden wird die Lösung mit ihren Komponenten und deren Interaktion dargestellt.
		\subsection{Komponenten}
			\begin{itemize}
				\item Frontend: In Angular geschrieben. -> Routing, HttpClient
				\item Filehandler: Springboot Mikroservice.
				\item Userhandler: Springboot Mikroservice. Generiert JWT.
				\item Datenbank: MariaDB. Vorteile Partitionierung
				\item Redis
				\item MinIO
			\end{itemize}
		\subsection{Interaktion Komponenten}
			\begin{figure}[h]
				\centering
				\includegraphics[width=0.8\linewidth]{download\_sequenz\_diagramm}
				\caption{Sequenz Diagramm für Download}
				\label{fig:partitionierung}
			\end{figure}
			
			\begin{figure}[h]
				\centering
				\includegraphics[width=0.8\linewidth]{rest\_sequenz\_diagramm}
				\caption{Sequenz Diagramm für restliche Funktionen}
				\label{fig:rest_sequenz_diagramm}
			\end{figure}
		
	\section{Anforderungen}
		Im folgenden sind die Funktionalen sowie die Nichtfunktionalen Anforderungen aufgelistet.
		\subsection{Funktional}
			\begin{itemize}
				\item \textbf{Datei-Upload}: Benutzer können Dateien in verschiedenen Formaten und Größen über eine Benutzeroberfläche oder API hochladen.
				\item \textbf{Tauschpartner}: Benutzer können einen anderen Benutzer hinzufügen und Dateien untereinander austauschen.
				\item \textbf{Load Balancing}: Das System verteilt eingehende Uploads und Anfragen automatisch auf verfügbare Server, um eine gleichmäßige Auslastung sicherzustellen.
				\item \textbf{Speicherung im Object Storage}: Dateien werden in einem verteilten Object-Storage-System gespeichert, das hohe Skalierbarkeit und Zuverlässigkeit bietet.
			\end{itemize}
		\subsection{Nichtfunktional}
			\begin{itemize}
				\item \textbf{Skalierbarkeit}: Das System muss in der Lage sein, mit einer wachsenden Anzahl von Benutzern und Dateien ohne Leistungseinbußen zu skalieren.
				\item \textbf{Kompatibilität}: Die Plattform muss mit gängigen Betriebssystemen (Windows, macOS, Linux) und Browsern kompatibel sein.
			\end{itemize}

\chapter{Umsetzung}
	Im Folgenden werden die Umsetzung und mögliche Alternativen dargestellt.
	\section{Implementierung}
		Bei der Implementierung schauen wir uns an, wie diese umgesetzt wurde und welche Schwierigkeiten es dabei gab und wie sie gelöst wurden.
		\subsection{Wie?}
			Docker compose 
			
			In der Docker compose werden zwei Shards definiert wobei jede 3 Minio Instanzen hat \ref{fig:shards}. 
		
			\begin{figure}[h]
				\centering
				\includegraphics[width=0.4\linewidth]{shards}
				\caption{Shards}
				\label{fig:shards}
			\end{figure}
			
			Partitionierung \ref{fig:partitionierung}.
			
			\begin{figure}[h]
				\centering
				\includegraphics[width=0.8\linewidth]{partitionierung}
				\caption{Partitionierung}
				\label{fig:partitionierung}
			\end{figure}
			
			Redis.
			
			k+1 
			
			Ein Bucket wenn 2 Personen sich adden
			
			\subsection{Schwierigkeiten}
				Folgenden Schwierigkeiten haben Stattgefunden und wie sie gelöst wurden.
				
				\subsubsection{Loadbalacing}
					Die erste Schwierigkeit war das Loadbalancen von den MinIO instanzen, Es war schwierig die Instanzen miteinander kommunizieren zu lassen. Um das zu lösen haben wir uns für eine andere Methode entschieden, und zwar zwei Shards zu kreieren die jeweils drei MinIO Instanzen beinhalten wobei in einer Shard alle 3 MinIO Instanzen das gleiche beinhalten. Somit wenn eine Ausfällt bleiben noch 2k andere.
			
				\subsubsection{Schardaustausch}
					Die zweite Schwierigkeit war der Austausch zwischen den beide Shards. Denn es kam zu Fehler mitunter den Loadbalancen. Um zu wissen auf welchen Shard gespeichert werden muss, musste einen Counter implementiert werden. Das wurde mit Redis gelöst.
				
				\subsubsection{Buckets zwischen zwei Benutzer}
					Die dritte Schwierigkeit war, dass wir entschieden haben, dass zwei Benutzer miteinander privat austaushen können. Bei einen einfachen Plattform wo es nur ein Bucket gibt wie Dropbox muss man nicht spezifische Buckets kreieren für 2 Benutzer. Bei Bucket kreieren muss man auf Namen achten, man darf ihn nur maximal 63 Zeichen geben. Um das zu lösen haben wir check Frontend und Backend Benutzername Maximal 8 Zeichen. und Bucketname ist user1-user2-bucket.
			
	\section{Mögliche Alternativen}
		Serverless Architektur
		% Eine serverlose Architektur mit Diensten wie AWS Lambda, Azure Functions oder Google Cloud Functions hätte die Möglichkeit geboten, nur dann Ressourcen zu nutzen, wenn spezifische Aktionen erforderlich sind (z. B. Datei-Upload, Benutzermanagement).
	
\chapter{Reflektion}
	\section{Was kann man anders machen?}
	\section{Größten Herausforderungen und ihre Lösung}
	Die größte Herausforderungen war das Zusammenspielen der verschiedenen Shards mit den MinIO Instanzen, der Datenbank und dem Loadbalancer.

%********************************
%Literaturverzeichnis
%********************************
\newpage
\pagenumbering{Roman}
\setcounter{page}{\value{frontmatterPage}} %Bei \pagenumbering wird Seitenzähler zurückgesetzt, hier wird der gespeicherte Wert vom frontmatter weitergeführt
\addtocounter{page}{1}
\addcontentsline{toc}{chapter}{\protect\numberline{}Literaturverzeichnis}

%Quellenverzeichnis
\renewcommand{\refname}{Literaturverzeichnis}
\bibliographystyle{IEEEtran}
\bibliography{./sources}


\end{document}