% !TeX program = xelatex
% ↑ Automatische Auswahl für XeLaTeX compiler

% Das ist mein Template für die TX000 Arbeiten. Nicht perfekt, also falls ihr Verbesserungsvorschläge habt, stellt gerne einen Pull-Request. https://github.com/NikomitK/TX000_Template
% Bitte lasst auch einen star auf github da, danke.
% Wenn ihr die cite funktion von LaTeX nutzen wollt, müsst ihr einfach die Quellen im bibtex Format in die sources.bib datei kopieren, google scholar z.B. hat bei den Quellen einen Button mit dem man das so bekommt, auch viele andere websites.
% Bilder kommen in den images Ordner, den müsst ihr beim abrufen eines Bildes nicht angeben, passiert automatisch.


% Trag hier deine Daten ein, die entsprechenden Felder werden automatisch angepasst.
\def\meinTitel{Verteilte Systeme: Filesharing-Plattform}
\def\artDerArbeit{Ausarbeitung}
\def\meinName{Aziz Carducci, Sven Sendke}
\def\meinKurs{TINF22F}
\def\meineMatrikelNr{1965015, 8469950}
\def\firmenName{Allianz Lebensversicherungs-AG}
\def\projektBetreuer{Benjamin Salchow}
\def\abgabeDatum{16.12.2024}
%-----------------------------------------------------------------------------------


\documentclass[12pt]{report}
\usepackage[heightrounded]{geometry}
\geometry{
	a4paper,
	lmargin=3.1cm, %Seitenrand left
	tmargin=2.8cm, %Seitenrad top
	headsep=35pt %Abstand von Kopfzeile
}
\usepackage[onehalfspacing]{setspace}
\usepackage[compact]{titlesec}
\usepackage{cite} % Zitierungen
\usepackage{struktex} % Struktogramme
\usepackage{array} % kp hat chatgpt benutzt
\usepackage{longtable} % Tabelle über einen seitenumbruch
\usepackage[nohyperlinks, printonlyused]{acronym} % abkürzungsverzeichnis
\usepackage{fontspec}
\usepackage{blindtext} % LoremIpsum
\usepackage{fancyhdr} % Kopf- und Fußzeile
\usepackage[export]{adjustbox} % Bilder alignment
\usepackage{stfloats} % Tabular at bottom
%\usepackage[table]{xcolor} %Tabellen farben
\usepackage{tabularx} % Anpassbarere tabellen
\usepackage{easyReview} % Review anmerkungen

\usepackage[utf8]{inputenc} % this is needed for umlauts
\usepackage[ngerman]{babel} % this is needed for umlauts
\usepackage[T1]{fontenc}    % this is needed for correct output of umlauts in pdf


\usepackage{subcaption} % Für subfigures glaub


\usepackage{tikz} % Zum zeichnen
\usetikzlibrary{calc}
\usetikzlibrary{shapes.geometric, arrows}
\setmainfont[Scale=1.1]{Arial}

%--------------------Flowcharts--------------------
\tikzstyle{startstop} = [rectangle, rounded corners, minimum width=3cm, minimum height=1cm,text centered, draw=black, fill=red!30]
\tikzstyle{io} = [trapezium, trapezium left angle=70, trapezium right angle=110, minimum width=3cm, minimum height=1cm, text centered, draw=black, fill=blue!30]
\tikzstyle{process} = [rectangle, minimum width=3cm, minimum height=1cm, text centered, draw=black, fill=orange!30]
\tikzstyle{decision} = [diamond, minimum width=3cm, minimum height=1cm, text centered, draw=black, fill=green!30]
\tikzstyle{arrow} = [thick,->,>=stealth]
%--------------------------------------------------


%--------------------Codeblöcke--------------------
\usepackage{listings} %Für Codeblöcke
\usepackage{color} %Farben für Codeblöcke?
\definecolor{dkgreen}{rgb}{0,0.6,0}
\definecolor{gray}{rgb}{0.5,0.5,0.5}
\definecolor{mauve}{rgb}{0.58,0,0.82}

\lstset{
	language=Java,
	aboveskip=3mm,
	belowskip=3mm,
	showstringspaces=false,
	columns=flexible,
	basicstyle={\small\ttfamily},
	numbers=none,
	numberstyle=\tiny\color{gray},
	keywordstyle=\color{blue},
	commentstyle=\color{dkgreen},
	stringstyle=\color{mauve},
	breaklines=true,
	breakatwhitespace=true,
	tabsize=3
}
%-------------------------------------------------

\usepackage{graphicx} %Package für Bilder
\graphicspath{ {./images/} } %Ordner für Bilder

\sloppy % damit lange Wörter nicht über die Zeile hinausgeschrieben werden.

%--------------------Chapter Heading--------------------
\makeatletter
\def\@makechapterhead#1{%
	\vspace*{-20\p@}%
	{\parindent \z@ \raggedright \normalfont
		\ifnum \c@secnumdepth >\m@ne
		%\huge\bfseries \@chapapp\space \thechapter
		\Huge\bfseries \thechapter.\space%
		%\par\nobreak
		%\vskip 20\p@
		\fi
		\interlinepenalty\@M
		\Huge \bfseries #1\par\nobreak
		\vskip 20\p@
}}
\makeatother
\makeatletter
\def\@makeschapterhead#1{%
	\vspace*{-20\p@}%
	{\parindent \z@ \raggedright \normalfont
		%\huge\bfseries \@chapapp\space \thechapter
		\Huge\bfseries\space%
		%\par\nobreak
		%\vskip 20\p@
		\interlinepenalty\@M
		\Huge \bfseries #1\par\nobreak
		\vskip 20\p@
}}
\makeatother
%-------------------------------------------------------

%\addto\captionsngerman{\renewcommand{\listfigurename}{}}
 % titel von abbildungsverzeichnis weg?

\setlength\parindent{0pt} %Auto Einrücken deaktivieren


%-------------Setup für Inhaltsverzeichnis--------------
\renewcommand{\contentsname}{Inhaltsverzeichnis} %Umbenennung TOC

\usepackage{tocloft} % Formatierung TOX

\setlength{\cftbeforetoctitleskip}{0pt}

\setlength{\cftsubsecnumwidth}{4em} %abstand kapitelnummer - titel

\renewcommand{\cfttoctitlefont}{\huge\bfseries}
\renewcommand{\cftloftitlefont}{\huge\bfseries}

\renewcommand\cftchapfont{\Large\bfseries}
\renewcommand\cftchappagefont{\large}

\renewcommand\cftsecfont{\large\bfseries}
\renewcommand\cftsecpagefont{\large}

\renewcommand\cftsubsecfont{\large}
\renewcommand\cftsubsecpagefont{\large}

\renewcommand\cftsubsubsecfont{\normalsize}
\renewcommand\cftsubsubsecpagefont{\normalsize}


\renewcommand\cftchapafterpnum{\par\addvspace{8pt}}
\renewcommand\cftsecafterpnum{\par\addvspace{8pt}}
\renewcommand\cftsubsecafterpnum{\par\addvspace{6pt}}
\renewcommand\cftsubsubsecafterpnum{\par\addvspace{6pt}}
%-------------------------------------------------------


%------------------Setup für LoF/LoT--------------------
\makeatletter
\renewcommand{\@cftmakeloftitle}{}
\renewcommand{\@cftmakelottitle}{}
\makeatother
\setlength{\cftfigindent}{0em} % change indentation of e.g. "Figure 1" within list of figures
\renewcommand\cftfigfont{\large}
\renewcommand\cftfigpagefont{\large}
\setlength{\cfttabindent}{0em} % change indentation of e.g. "Figure 1" within list of figures
\renewcommand\cfttabfont{\large}
\renewcommand\cfttabpagefont{\large}
\setlength{\cftbeforeloftitleskip}{0pt}
\setlength{\cftbeforelottitleskip}{0pt}
%-------------------------------------------------------


%----------------Setup für Verlinkungen-----------------
\usepackage{hyperref}
\hypersetup{
	colorlinks,
	citecolor=black,
	filecolor=black,
	linkcolor=black,
	urlcolor=black
}
%-------------------------------------------------------


%-------------Kopf-/Fußzeile für Titlepage--------------
\fancypagestyle{titlepage}
{
	\fancyhead[L]{\includegraphics[scale=0.09]{firmenlogo}}
	\fancyhead[R]{\includegraphics[scale=0.25]{dhbw}}
	\renewcommand{\headrulewidth}{0pt}
	\fancyfoot[C]{}
}
%-------------------------------------------------------

\begin{document} 
	
	\begin{titlepage}
		\thispagestyle{titlepage}
		\newcommand\HRule{\rule{\textwidth}{1pt}} %Titellinien
		
		
		\begin{center}
			
			\vspace*{2cm}
			
			%Title
			\begin{spacing}{2}
				{ \huge \bfseries \MakeUppercase{\meinTitel}}
				%{ \large \bfseries subTitle}\\[0.4cm]
			\end{spacing}
			
			\vspace*{1.5cm}
			
			%Art der Arbeit
			\Large \artDerArbeit
			
			\vspace*{2cm}
			
			%Hochschule
			{\LARGE Studiengang Informatik}\\
			{\LARGE an der Dualen Hochschule}\\
			{\LARGE Baden-Württemberg Stuttgart}\\
			
			\vspace*{2cm}
			
			\Large von \meinName
			
			\vspace*{1.5cm}
			
			\Large Abgabedatum: \abgabeDatum
			
			\begin{table*}[bp]
				\centering
				\begin{tabular}{l l}
					Kurs: & \meinKurs  \\
					Matrikelnummer: & \meineMatrikelNr  \\
					Unternehmen: & \firmenName \\
					Projektbetreuer: &  \projektBetreuer\\
				\end{tabular}
			\end{table*}
			
			
		\end{center}
		
	\end{titlepage}


%------------------Kopf- und Fußzeile-------------------
\fancypagestyle{plain}{
	\fancyfoot[L]{\meinName\\
		 \meinKurs}
	\fancyfoot[C]{Seite \thepage\ }% von \pageref{LastPage}}
	\fancyfoot[R]{\abgabeDatum}
}

\pagestyle{plain}
\fancyhead{}


\fancyhead[L]{\includegraphics[scale=0.09]{firmenlogo}}
\fancyhead[C]{\nouppercase\leftmark}
\fancyhead[R]{\includegraphics[scale=0.25]{dhbw}}

\renewcommand{\footrulewidth}{0.4pt} %Linie für Fußzeile


\renewcommand{\sectionmark}[1]{\markboth{#1}{}} 
%-------------------------------------------------------

\pagenumbering{Roman}
\newpage

%------------------Inhaltsverzeichnis-------------------

\addcontentsline{toc}{chapter}{\protect\numberline{}Inhaltsverzeichnis}
\tableofcontents
\addtocontents{toc}{}
\thispagestyle{plain}

%********************************
%Abbildungsverzeichnis
%********************************
\newpage
\chapter*{Abbildungsverzeichnis}
\addcontentsline{toc}{chapter}{\protect\numberline{}Abbildungsverzeichnis}

\listoffigures

%Speichern des page counters, um bei Literaturverzeichnis weiter zu zählen.
\newcounter{frontmatterPage}
\addtocounter{frontmatterPage}{\value{page}} 

\newpage
\pagenumbering{arabic}
\chapter{Architektur}
	Im Folgenden wird die Architektur im Allgemeinen mit den Systemkomponenten und den Anforderungen dargestellt.
	\section{Allgemein}
		Zuerst betrachten wir die Struktur der Architektur und dann die Architectual Desicion.
		\subsection{Aufbau}
			Die Abbildung \ref{fig:architektur} zeigt die Gesamtarchitektur. Auf die einzelnen Aspekte wird in den folgenden Punkten eingegangen.
			\begin{figure}[h]
				\centering
				\includegraphics[width=\linewidth]{architektur}
				\caption{Architektur}
				\label{fig:architektur}
			\end{figure}
			
		\subsection{Architectual Decision}
			Folgende Architectual Decision wurden getroffen.
			
			\subsubsection{Microservice-Architektur}
			Die Entscheidung für eine Microservice-Architektur wurde getroffen, um die Modularisierung des Systems zu fördern. Durch die Aufteilung in einzelne, voneinander unabhängige Komponenten wird eine flexible Skalierung ermöglicht, wobei beispielsweise der Filehandler stärker skaliert werden kann als der Userhandler. Zudem erhöht diese Architektur die Ausfallsicherheit, da fehlerhafte Microservices schnell neugestartet werden können, ohne das gesamte System zu beeinträchtigen \cite{taibi2017processes}. Ein weiterer zentraler Vorteil ist die hohe Wartbarkeit: Die Aufteilung in unabhängige, selbstständig deploybare Services erleichtert es Entwicklerteams, Änderungen und Tests durchzuführen, ohne andere Teile des Systems zu beeinflussen, was die verteilte Entwicklung vereinfacht \cite{de2019monolithic}.
			
			\subsubsection{MinIO}
			MinIO wurde als Objektspeicherlösung gewählt, da es die Hochleistungsanforderungen moderner Big-Data-Anwendungen erfüllt \cite{makris2022performance}. MinIO bietet sowohl bei Lese- als auch bei Schreiboperationen eine herausragende Leistung \cite{makris2022performance}. Wir haben uns für eine S3-Kompatible Objektspeicher-Server entschieden, weil es eine der weit verbreitesten Objektspeicherungs Platforme ist und welche auch gut dokumentiert ist, aber unteranderem auch da es bei sehr vielen großen Unternehmen genutzt wird und wir diese Erfahrung mitnehmen wollten.
			
			\subsubsection{Spring Boot}
			Das Spring Boot Framework ist der De-facto-Standard für Java-Microservice-Architekturen. Es zeichnet sich durch seine Fähigkeit aus, enorme Datenmengen zu verarbeiten, was es besonders für die Big-Data-Industrie attraktiv macht \cite{mythily2022analysis}. Darüber hinaus erleichtert Spring Boot die Entwicklung von RESTful-Webservices und APIs erheblich, wodurch Entwickler ihre Aufgaben schneller und effizienter erledigen können \cite{mythily2022analysis}.
			
			\subsubsection{Relationale Datenbank}
			Der Vorteil der relationalen Datenbank besteht darin, dass sie eine solide Grundlage für die Behandlung von Redundanz und Konsistenz von Beziehungen bietet. \cite{codd1970relational}. Dies ist wichtig, um die Metadaten der Dateien konsistent zu halten. 
			
	\section{Systemkomponenten}
		Im Folgenden wird die Lösung mit ihren Komponenten und deren Interaktion dargestellt.
		\subsection{Komponenten}
			\begin{itemize}
				\item Frontend (Angular): Angular erleichtert sowohl das Routing durch Angular Router als auch REST durch den Angular Service HttpClient.
				\item Filehandler (Spring Boot): Dieser Microservice kümmert sich um die Benutzerregistrierung und -anmeldung und stellt den JWT bereit.
				\item Userhandler (Spring Boot): Dieser Microservice kümmert sich um alle Funktionen, die mit Dateien zu tun haben.
				\item Datenbank (MariaDB):  Speichert Benutzer, Datei-Metadaten und begonnene Konversationen zwischen 2 Benutzern. 
				\item Redis: Dient als Cache für den Filehandler, mehr dazu später.
				\item Objektspeicherung (MinIO): Drei Minio Instanzen befinden sich in einem Shard. In einem Shard speichern sie dasselbe, aber ansonsten wechseln sie zwischen den Shards, um die Datei zu speichern.
				\item Loadbalancer (Nginx): Verteilt alle Anfragen des Benutzers.
			\end{itemize}
			
		\subsection{Interaktion Komponenten}
			Die Interaktion zwischen den Komponenten ist in den folgenden Sequenzdiagrammen dargestellt, wobei die Upload-Funktion in Abbildung \ref{fig:upload_sequenz_diagramm} und die anderen Funktionen wie Benutzerregistrierung und -anmeldung , Bucket-Erstellung, Löschen/Download von Dateien in Abbildung \ref{fig:rest_sequenz_diagramm} dargestellt sind. Die Entscheidung Upload-File einzeln darzustellen, war weil es unsere Hauptfunktion ist, wo auch am meisten passiert.
			
			\begin{figure}[h]
				\centering
				\includegraphics[width=0.95\linewidth]{upload\_sequenz\_diagramm}
				\caption{Sequenz Diagramm für Upload}
				\label{fig:upload_sequenz_diagramm}
			\end{figure}
			
			\begin{figure}[h]
				\centering
				\includegraphics[width=0.95\linewidth]{rest\_sequenz\_diagramm}
				\caption{Sequenz Diagramm für restliche Funktionen}
				\label{fig:rest_sequenz_diagramm}
			\end{figure}
		
	\section{Anforderungen}
		Im folgenden sind die Funktionalen sowie die Nichtfunktionalen Anforderungen aufgelistet.
		\subsection{Funktional}
			\begin{itemize}
				\item \textbf{Datei-Upload}: Benutzer können Dateien in verschiedenen Formaten über eine Benutzeroberfläche hochladen.
				\item \textbf{Tauschpartner}: Benutzer können einen anderen Benutzer hinzufügen und Dateien untereinander austauschen.
				\item \textbf{Load Balancing}: Das System verteilt eingehende Uploads und Anfragen automatisch auf verfügbare Server, um eine gleichmäßige Auslastung sicherzustellen.
				\item \textbf{Speicherung im Object Storage}: Dateien werden in einem verteilten Object-Storage-System gespeichert, das hohe Skalierbarkeit und Zuverlässigkeit bietet.
			\end{itemize}
		\subsection{Nichtfunktional}
			\begin{itemize}
				\item \textbf{Skalierbarkeit}: Das System muss in der Lage sein, mit einer wachsenden Anzahl von Benutzern und Dateien ohne Leistungseinbußen zu skalieren.
				\item \textbf{Kompatibilität}: Die Plattform muss mit gängigen Betriebssystemen (Windows, macOS, Linux) und Browsern kompatibel sein.
				\item \textbf{Ausfallsicherheit}: Das System muss bei einem Ausfall einzelner Komponenten wie MinIO-Instanzen oder Servern weiterhin stabil funktionieren und den Betrieb fortsetzen.
			\end{itemize}

\chapter{Umsetzung}
	Im Folgenden werden die Umsetzung und mögliche Alternativen dargestellt.
	\section{Implementierung}
		Bei der Implementierung schauen wir uns an, wie diese umgesetzt wurde und welche Schwierigkeiten es dabei gab und wie sie gelöst wurden.
		\subsection{Wie wird die Architektur umgesetzt?}
			\subsubsection{Docker Compose}
				Die Architektur wird mit Docker Compose erstellt. Das Frontend, die Microservices, die Datenbank, der Loadbalancer, Redis und die Shards mit den MinIO-Instanzen werden mit den entsprechenden Dockerfiles gebaut. Die sich daraus ergebende Shard-Struktur ist in Abbildung \ref{fig:shards} dargestellt. Wie zu sehen ist, wird das Prinzip mit 2k+1 für die Ausfallsicherheit eingehalten, auch werden in unserem Beispiel jeweils drei Microservices gestartet.
		
				\begin{figure}[h]
					\centering
					\includegraphics[width=0.8\linewidth]{shards}
					\caption{Shards}
					\label{fig:shards}
				\end{figure}
			
			\subsubsection{Partitionierung}
				Wir haben die horizontale Partitionierung in unserem Projekt implementiert, um die Datenbank effizienter zu machen \cite{alsultanny2010database}. Dabei teilen wir die Daten nach Wochen auf, so dass jede Woche ihre eigene Partition hat. Dies hat den Vorteil, dass bei Abfragen nur die relevanten Daten der angefragten Woche geladen werden müssen, was die Abfragegeschwindigkeit erhöht und die Gesamtzahl der Lesevorgänge verringert. Dies sorgt für eine bessere Performance, insbesondere bei einem skalierten Projekt, da durch das Partnersystem bis zu Hunderte von Dateien pro Woche gesendet werden können, was schnell dazu führt, dass die Datenbank sehr voll wird. Die Partitionierung ist in der Abbildung \ref{fig:partitionierung} dargestellt.
				
				\begin{figure}[h]
					\centering
					\includegraphics[width=0.8\linewidth]{partitionierung}
					\caption{Partitionierung}
					\label{fig:partitionierung}
				\end{figure}
				
			\subsubsection{Redis}
				Die Implementierung von Redis ermöglicht eine globale, persistente \cite{eddelbuettel2022brief} Speicherung des Round-Robin-Zählers (Caching), der über verschiedene Backend-Instanzen des Service synchronisiert wird. Dies sorgt für eine konsistente Lastverteilung zwischen den Minio-Instanzen und verbessert gleichzeitig die Skalierbarkeit und Fehlertoleranz der Anwendung. Dies ermöglicht einen globalen Round-Robin Service, wenn mehr als ein Server aktiv ist.
				
			\subsubsection{Buckets}
				Es war nicht das Ziel, ein Dropbox-ähnliches Filesharing-System zu schaffen, da ein solcher Ansatz nur einen einzigen Bucket benötigt, in dem alle Dateien gespeichert werden. Dies wäre zwar technisch möglich, aber die einfachere Lösung. Stattdessen wollten wir uns intensiver mit der Funktionsweise von Buckets und dem S3-Modell beschäftigen. So haben wir uns entschieden, dass zwei Benutzer sich gegenseitig hinzufügen können und für den Austausch ihrer Dateien ein separater Bucket angelegt wird, so dass die Dateien nur für diese beiden zugänglich sind. Auf diese Weise konnten wir nicht nur die Erstellung von Buckets genauer untersuchen, sondern auch eine Verbindung zwischen den einzelnen Dateien und den Buckets herstellen.
				
			\subsubsection{Ausfallsicherheit}
				Um die Ausfallsicherheit zu gewährleisten, wurden mehrere Vorkehrungen getroffen. Erstens, wenn eine MinIO-Instanz in einem Shard ausfällt, laufen die anderen beiden normal weiter und können weiter verwendet werden. Außerdem wird ein Healthcheck im Code gemacht, wenn alle MinIO-Instanzen eines Shards ausfallen, also der Shard down ist, kommt eine Info-Meldung und die Datei wird in der anderen Shard hinzugefügt. Das bedeutet, dass alle Instanzen bis auf eine MinIO-Instanz ausfallen können und das Programm normal weiterlaufen würde. Wenn das Frontend eine Anfrage für eine Datei macht, die nicht mehr existiert, weil ein Shard ausgefallen ist, dann wird diese im Frontend gelöscht, wenn mit dieser Datei interagiert wird.
				
			\subsection{Schwierigkeiten}
				Folgenden Schwierigkeiten haben Stattgefunden und wie sie gelöst wurden.
				
				\subsubsection{Loadbalacing}
					Die erste Schwierigkeit war das Loadbalancen von den MinIO-Instanzen, hierbei war es schwierig die einzelnen MinIO-Instanzen über nginx kommunizieren zu lassen, dies wurde gelöst in dem die MinIO-Instanzen jetzt direkt auf einander zugreifen, durch das Driver-System.
			
				\subsubsection{Schardaustausch}
					Die zweite Schwierigkeit war der Austausch zwischen den beiden Shards. Dies lag daran, dass es manchmal zu Fehlern beim Loadbalancing kam. Um zu wissen auf welchem Shard gespeichert werden soll, musste ein Counter implementiert werden, hier war aber das Problem das jede Backend-Instanz einen eigenen Counter initialisiert hat, um dies zu lösen wurde mit Redis der Counter im Cache gespeichert.
				
				\subsubsection{Buckets zwischen zwei Benutzer}
					Die dritte Schwierigkeit war, dass wir uns dafür entschieden haben, dass 2 Benutzer Dateien privat austauschen können. Bei einer einfachen Plattform, wo es nur einen Bucket gibt, wie bei Dropbox, ist es nicht notwendig, für 2 Benutzer Buckets zu erstellen. Bei der Erstellung von Buckets muss man auf die Namen achten, man kann nur maximal 63 Zeichen vergeben. Um dieses Problem zu lösen, geben wir im Frontend und im Backend Benutzername eine maximale Länge von 8 Zeichen und indem der Bucketname user1-user2-bucket ist.
			
	\section{Mögliche Alternativen}
		Obwohl es natürlich Alternativen in der Programmierung gibt, wie z.B. weniger oder mehr Metadaten für die verschiedenen benötigten Informationen zu speichern, wäre eine Alternative in größerem Umfang andere Open Source S3-kompatible Software wie Ceph oder GlusterFS zu verwenden. Anstelle von S3 könnte auch Hadoop HDFS (Hadoop Distributed File System) verwendet werden. Immer mehr Anwendungen nutzen Hadoop-Cluster, da Organisationen ein einfaches und effizientes Modell entdeckt haben, das hervorragend in verteilten Umgebungen funktioniert. \cite{ghazi2015hadoop} Dieses Modell wird vor allem im Bereich Big Data und Analytics eingesetzt, ist jedoch im Gegensatz zu den anderen genannten Lösungen nicht S3-kompatibel.
	
\chapter{Reflektion}
	Die folgenden Überlegungen zeigen, was wir anders machen können und was die größten Herausforderungen und ihre Lösungen sind.
	
	\section{Was kann man anders machen?}
	Ein entscheidender Lernpunkt während der Entwicklung war das Datenbanksystem. Rückblickend wäre es sinnvoll gewesen, dieses von Anfang an korrekt zu konzipieren und aufzusetzen. Stattdessen mussten wir es mehrfach anpassen und nachbessern, bis es schließlich wie gewünscht funktionierte. Besonders wichtig wäre es gewesen, von Beginn an auf Skalierbarkeit und die Einhaltung etablierter Konventionen zu achten. Dies hätte nicht nur spätere Komplikationen vermieden, sondern auch die Integration und Wartung des Systems erheblich erleichtert. Dieser Aspekt unterstreicht, wie entscheidend ein solides Fundament bei der Entwicklung komplexer Systeme ist.
	
	\section{Größten Herausforderungen und ihre Lösung}
	Die größte Herausforderung lag in der Koordination und Integration der verschiedenen Systemkomponenten, insbesondere der Shards mit den MinIO-Instanzen, der Datenbank und dem Loadbalancer. Anfangs scheiterte die Kommunikation zwischen den Instanzen, was dazu führte, dass Dateien nicht korrekt gespeichert oder geladen wurden. Dieses Problem lösten wir durch die Einführung eines zweistufigen Shard-Systems, bei dem jeweils drei MinIO-Instanzen innerhalb eines Shards die gleichen Daten speichern. Dieses Redundanzkonzept gewährleistet die Ausfallsicherheit und ermöglicht den reibungslosen Betrieb, selbst wenn eine Instanz ausfällt.
	
	Ein weiteres Problem bestand im Austausch zwischen den beiden Shards. Hier traten Fehler auf, die durch inkorrekte Loadbalancing-Mechanismen verursacht wurden. Um zu bestimmen, welcher Shard für die Speicherung verwendet werden sollte, führten wir einen Zähler ein, der mit Redis implementiert wurde. Redis diente dabei als schneller und zuverlässiger Speicher für den aktuellen Status des Counters.

%********************************
%Literaturverzeichnis
%********************************
\newpage
\pagenumbering{Roman}
\setcounter{page}{\value{frontmatterPage}} %Bei \pagenumbering wird Seitenzähler zurückgesetzt, hier wird der gespeicherte Wert vom frontmatter weitergeführt
\addtocounter{page}{1}
\addcontentsline{toc}{chapter}{\protect\numberline{}Literaturverzeichnis}

%Quellenverzeichnis
\renewcommand{\refname}{Literaturverzeichnis}
\bibliographystyle{/usr/local/texlive/2024/texmf-dist/bibtex/bst/ieeetran/IEEEtran.bst}
\bibliography{./sources}


\end{document}